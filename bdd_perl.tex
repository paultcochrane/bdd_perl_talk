\documentclass[10pt]{vcs_beamer}

\newcommand{\pending}[1]{\color{DarkOrange}{#1}}
\newcommand{\scenario}[1]{\color{RoyalBlue}{#1}}
\newcommand{\str}[1]{\color{Turquoise}{#1}}
\newcommand{\error}[1]{\color{red}{#1}}
\newcommand{\pass}[1]{\color{ForestGreen}{#1}}

\begin{document}

\author{Paul Cochrane}
\title{Adding pherkin to your diet}
\subtitle{Behaviour Driven Development in Perl}
\date{\today}

\maketitle

%%%%%%%%%%%%%%%%%%%%%%%%%%%%%%%%%%%%%%%%%%%%%%%%%%%%%%%%%%%%%%%%%%%%%%%%%%%%
\begin{frame}{What is Behaviour Driven Development?}

\begin{itemize}
    \item Comes from the Agile concept of \emph{User Stories}
    \item A form of requirements specification
    \item Outside-in development
    \item Complements and uses TDD
    \item Not a replacement for TDD!
\end{itemize}

BDD: user stories -> high level features -> low level functionality

TDD: low level functionality -> high level features

\end{frame}

%%%%%%%%%%%%%%%%%%%%%%%%%%%%%%%%%%%%%%%%%%%%%%%%%%%%%%%%%%%%%%%%%%%%%%%%%%%%
\begin{frame}[fragile]
    \frametitle{Cucumber}

\begin{itemize}
    \item Stakeholders/Domain experts write requirements in business-readable
        domain-specific language.
    \item This can be parsed and the underlying tests get executed.
    \item Serves as documentation for the desired features.
    \item Domain experts don't have to write code!
    \item Parseable natural language: \emph{Gherkin}
\end{itemize}

\begin{lstlisting}[escapeinside='']
'\color{BlueViolet}{Feature:}' User login
    As a user of the system
    In order to see my account information
    I want to be able to log in

    '\color{BlueViolet}{Scenario:}' Successful login
        '\color{DarkOrange}{Given}' I have an account on the system
        '\color{RoyalBlue}{When}' I enter my username and password
        '\color{ForestGreen}{Then}' I should be presented with my account information
\end{lstlisting}
\end{frame}

%%%%%%%%%%%%%%%%%%%%%%%%%%%%%%%%%%%%%%%%%%%%%%%%%%%%%%%%%%%%%%%%%%%%%%%%%%%%
\begin{frame}{Cucumber history}

\begin{itemize}
    \item Java $\rightarrow$ jbehave, now Cuke for Duke
    \item .NET $\rightarrow$ Cuke for Nuke
    \item Ruby $\rightarrow$ rbehave $\rightarrow$ cucumber
    \item Python $\rightarrow$ behave
    \item Perl $\rightarrow$ Test::BDD::Cucumber, pherkin
    \item More info from \url{cukes.info}
\end{itemize}

\end{frame}

%%%%%%%%%%%%%%%%%%%%%%%%%%%%%%%%%%%%%%%%%%%%%%%%%%%%%%%%%%%%%%%%%%%%%%%%%%%%
\begin{frame}
\frametitle{Gherkin the language}
\begin{description}
    \item[Feature] name of current feature
        \begin{description}
            \item[In order to] [business value]
            \item[As a] [role]
            \item[I want to] [some action]
        \end{description}
\end{description}

\begin{description}
    \item[Scenario] name of current scenario
        \begin{description}
            \item[Given] [context/state]
            \item[When] [action]
            \item[And] [another action] (optional)
            \item[But] [not some other action] (optional)
            \item[Then] [outcome]
            \item[And] [another outcome] (optional)
            \item[But] [not some other outcome] (optional)
        \end{description}
\end{description}
\end{frame}

%%%%%%%%%%%%%%%%%%%%%%%%%%%%%%%%%%%%%%%%%%%%%%%%%%%%%%%%%%%%%%%%%%%%%%%%%%%%
\begin{frame}[fragile]
\frametitle{Gherkin works in \emph{your} language}

\begin{lstlisting}[escapeinside='']
Funktionalit'ä't: dickes Auto fahren
    Als deutsche Autofahrer
    Damit ich schnell fahren kann
    Will ich ein dickes Auto haben

    Szenario: auf der Autobahn fahren
        Angenommen, dass ich auf der Autobahn fahre
        Wenn ich Gas gebe
        Dann soll ich mit mindestens 180 Sachen unterwegs sein
\end{lstlisting}


\begin{lstlisting}
OHAI: I CAN HAS WISHEZ PLZ
    SO DAT I CAN HAS HUGS TIEM AN PLAY TIEM AN SLEEP TIEM
    AS NICE GREEDEE KITTEH
    I CAN PLZ MANEGE TEH WISHEZ

    MISHUN: MAK NEW WISH
        GIVN I AM LOGGD IN
        WEN I MAK "CHEEZBURGER" WISH
        DEN I C "CHEEZBURGER" ON ME WISHEZ LIST
\end{lstlisting}

\end{frame}

%%%%%%%%%%%%%%%%%%%%%%%%%%%%%%%%%%%%%%%%%%%%%%%%%%%%%%%%%%%%%%%%%%%%%%%%%%%%
\begin{frame}{BDD development cycle}
    \begin{itemize}
        \item Start a feature
        \item Write a scenario describing an aspect of that feature
        \item Write the steps associated with that scenario
        \item Run \ttt{pherkin}
        \item Steps are undefined
        \item Write steps in unit-testing framework
        \item Steps fail
        \item Red $\rightarrow$ Green $\rightarrow$ Refactor until step passes
        \item Move on to next step
        \item Wash, rinse, repeat
    \end{itemize}
\end{frame}

%%%%%%%%%%%%%%%%%%%%%%%%%%%%%%%%%%%%%%%%%%%%%%%%%%%%%%%%%%%%%%%%%%%%%%%%%%%%
\begin{frame}[fragile]
\frametitle{Required directory structure for Cucumber}

\begin{lstlisting}
project/
    features/
        my_cool_feature.feature
        step_definitions/
            cool_steps.pl
            other_steps.pl
\end{lstlisting}

\begin{itemize}
    \item \ttt{.feature} file is written in Gherkin
    \item step files are written in Perl
    \item step files use your favourite testing framework (e.g.
        \ttt{Test::More})
\end{itemize}

\end{frame}

%%%%%%%%%%%%%%%%%%%%%%%%%%%%%%%%%%%%%%%%%%%%%%%%%%%%%%%%%%%%%%%%%%%%%%%%%%%%
\begin{frame}[fragile]
\frametitle{A program to say ``moin''}

Run \ttt{pherkin}
\begin{lstlisting}
$ pherkin
No feature files found at ....
\end{lstlisting}
%stopzone

Create a feature file and a \ttt{step\_definitions} directory
\begin{lstlisting}
$ mkdir -p features/step_definitions
$ vim features/say_moin.feature
Feature: Say moin to names in list
    As a user of the software
    In order to be a nice person
    I want to say "moin" to others

    Scenario: one name
        Given the name "bob"
        When I say moin
        Then I should see "moin bob"
\end{lstlisting}
%stopzone

\end{frame}

%%%%%%%%%%%%%%%%%%%%%%%%%%%%%%%%%%%%%%%%%%%%%%%%%%%%%%%%%%%%%%%%%%%%%%%%%%%%
\begin{frame}[fragile]
\frametitle{Example: A program to say ``moin''}

Running \ttt{pherkin} again should now show us boilerplate code we can copy
and paste to create our step definitions (standard behaviour of Cucumber and
Behave), however this isn't yet implemented.

\begin{itemize}
    \item Running \ttt{pherkin} shows us that all steps are pending.
\end{itemize}

\begin{lstlisting}[escapeinside='']
Scenario: '\color{RoyalBlue}{one name}'
  '\color{DarkOrange}{Given the name} \color{Turquoise}{"bob"}'
  '\color{DarkOrange}{When I say moin}'
  '\color{DarkOrange}{Then I should see} \color{Turquoise}{"moin bob"}'
\end{lstlisting}

\end{frame}

%%%%%%%%%%%%%%%%%%%%%%%%%%%%%%%%%%%%%%%%%%%%%%%%%%%%%%%%%%%%%%%%%%%%%%%%%%%%
\begin{frame}[fragile]
\frametitle{Example: A program to say ``moin''}

\begin{itemize}
    \item Create the step file
\end{itemize}

\begin{lstlisting}[language=perl]
use warnings;
use strict;

use Test::More;
use Test::BDD::Cucumber::StepFile;
use Method::Signatures;

Given qr//, func($c) {  # $c is an object holding context info
    ok(0);
};

When qr//, func($c) {
    ok(0);
};

Then qr//, func($c) {
    ok(0);
};
\end{lstlisting}
%stopzone

\end{frame}

%%%%%%%%%%%%%%%%%%%%%%%%%%%%%%%%%%%%%%%%%%%%%%%%%%%%%%%%%%%%%%%%%%%%%%%%%%%%
\begin{frame}[fragile]
\frametitle{Example: A program to say ``moin''}

\begin{itemize}
    \item \ttt{Given} step will fail.  \ttt{When} and \ttt{Then} steps are
        still pending.
\end{itemize}

\begin{lstlisting}[escapeinside='']
Scenario: '\scenario{one name}'
    '\error{Given the name}' '\str{"bob"}'
    '\error{step defined at features/say\_moin.feature line 7.}'
    '\error{ok 1 - Starting to execute step: the name "bob"}'
    '\error{not ok 2}'
    '\error{\#   Failed test at features/step\_definitions/say\_moin\_steps.pl line 16.}'
    '\error{1..2}'
  '\pending{When I say moin}'
  '\pending{Then I should see}' '\str{"moin bob"}'
\end{lstlisting}

\begin{itemize}
    \item Write \ttt{Given} test.
\end{itemize}

\begin{lstlisting}[language=perl]
Given qr/the name "(\w+)"/, func($c) {
    use MoinSayer;
    my $sayer = MoinSayer->new(name => $1);
    ok( $sayer, "Object created" );
    $c->stash->{'scenario'}->{'object'} = $sayer;
};
\end{lstlisting}

\end{frame}

%%%%%%%%%%%%%%%%%%%%%%%%%%%%%%%%%%%%%%%%%%%%%%%%%%%%%%%%%%%%%%%%%%%%%%%%%%%%
\begin{frame}[fragile]
\frametitle{Example: A program to say ``moin''}

\begin{itemize}
    \item \ttt{Given} step naturally still fails (but now for a different
        reason).
    \item Create \ttt{lib} dir, add \ttt{MoinSayer} class and \ttt{name}
        attribute.
\end{itemize}

\begin{lstlisting}[language=perl]
$ mkdir lib
$ vim lib/MoinSayer.pm
...
has 'name' => (
    is => 'ro',
    isa => 'Str',
);
...
\end{lstlisting}
%stopzone

\end{frame}

%%%%%%%%%%%%%%%%%%%%%%%%%%%%%%%%%%%%%%%%%%%%%%%%%%%%%%%%%%%%%%%%%%%%%%%%%%%%
\begin{frame}[fragile]
\frametitle{Example: A program to say ``moin''}

\begin{itemize}
    \item \ttt{Given} step now passes; \ttt{When} step now fails; \ttt{Then}
        step still pending.
\end{itemize}

\begin{lstlisting}[escapeinside='']
$ pherkin -I lib
Scenario: '\scenario{one name}'
  '\pass{Given the name} \str{"bob"}'
  '\error{When I say moin}'
    '\error{step defined at features/say\_moin.feature line 8.}'
    '\error{ok 1 - Starting to execute step: I say moin}'
    '\error{not ok 2}'
    '\error{\#   Failed test at features/step\_definitions/say\_moin\_steps.pl line 23.}'
    '\error{1..2}'
  '\pending{Then I should see} \str{"moin bob"}'
\end{lstlisting}
%stopzone

\begin{itemize}
    \item \ttt{pherkin -l} is the same as \ttt{pherkin -I lib}
\end{itemize}
\end{frame}

%%%%%%%%%%%%%%%%%%%%%%%%%%%%%%%%%%%%%%%%%%%%%%%%%%%%%%%%%%%%%%%%%%%%%%%%%%%%
\begin{frame}[fragile]
\frametitle{Example: A program to say ``moin''}

\begin{itemize}
    \item Write \ttt{When} test.
\end{itemize}

\begin{lstlisting}[language=perl]
...
my $stdout = capture_stdout {
    $c->stash->{'scenario'}->{'object'}->moin();
};
$c->stash->{'scenario'}->{'stdout'} = $stdout;
...
\end{lstlisting}

\begin{itemize}
    \item Write stub \ttt{moin()} method.
\end{itemize}

\begin{lstlisting}[language=perl]
...
sub moin {
    my $self = shift;
}
...
\end{lstlisting}
%stopzone

\end{frame}

%%%%%%%%%%%%%%%%%%%%%%%%%%%%%%%%%%%%%%%%%%%%%%%%%%%%%%%%%%%%%%%%%%%%%%%%%%%%
\begin{frame}[fragile]
\frametitle{Example: A program to say ``moin''}

\begin{itemize}
    \item \ttt{When} step now passes.  \ttt{Then} step now fails.
\end{itemize}

\begin{lstlisting}[escapeinside='']
Scenario: '\scenario{one name}'
  '\pass{Given the name} \str{"bob"}'
  '\pass{When I say moin}'
  '\error{Then I should see} \str{"moin bob"}'
    '\error{step defined at features/say\_moin.feature line 9.}'
    '\error{ok 1 - Starting to execute step: I should see "moin bob"}'
    '\error{not ok 2}'
    '\error{\#   Failed test at features/step\_definitions/say\_moin\_steps.pl line 28.}'
    '\error{1..2}'
\end{lstlisting}

\end{frame}

%%%%%%%%%%%%%%%%%%%%%%%%%%%%%%%%%%%%%%%%%%%%%%%%%%%%%%%%%%%%%%%%%%%%%%%%%%%%
\begin{frame}[fragile]
\frametitle{Example: A program to say ``moin''}

\begin{itemize}
    \item Implement \ttt{Then} test and \ttt{moin} sub.
\end{itemize}

\begin{lstlisting}[language=perl]
...
my $stdout = $c->stash->{'scenario'}->{'stdout'};
like( $stdout, qr/moin $1/, "Said expected greeting" );
...
\end{lstlisting}

\begin{lstlisting}[language=perl]
...
sub moin {
    my $self = shift;
    print "moin ", $self->name, "\n";
}
...
\end{lstlisting}
%stopzone

\end{frame}

%%%%%%%%%%%%%%%%%%%%%%%%%%%%%%%%%%%%%%%%%%%%%%%%%%%%%%%%%%%%%%%%%%%%%%%%%%%%
\begin{frame}[fragile]
\frametitle{Example: A program to say ``moin''}

\begin{itemize}
    \item All steps now pass.
\end{itemize}

\begin{lstlisting}[escapeinside='']
Scenario: '\scenario{one name}'
  '\pass{Given the name} \str{"bob"}'
  '\pass{When I say moin}'
  '\pass{Then I should see} \str{"moin bob"}'
\end{lstlisting}

\begin{itemize}
    \item Now add next scenario and slowly complete the feature
\end{itemize}

\end{frame}

%%%%%%%%%%%%%%%%%%%%%%%%%%%%%%%%%%%%%%%%%%%%%%%%%%%%%%%%%%%%%%%%%%%%%%%%%%%%
\begin{frame}[fragile]
\frametitle{Tables}

\begin{itemize}
    \item How to handle multiple inputs and outputs?
\end{itemize}

\begin{lstlisting}
Scenario: say moin to multiple users
  Given a MoinSayer object
  When I say "moin" to a "<user>"
  Then I see the message "<message>"
  Examples:
    | user  | message    |
    | bob   | moin bob   |
    | alice | moin alice |
    | eve   | moin eve   |
\end{lstlisting}

\end{frame}

%%%%%%%%%%%%%%%%%%%%%%%%%%%%%%%%%%%%%%%%%%%%%%%%%%%%%%%%%%%%%%%%%%%%%%%%%%%%
\begin{frame}[fragile]
\frametitle{Background information}

\begin{itemize}
    \item Sometimes scenarios have redundant information
    \item This can be extracted into \ttt{Background} information
\end{itemize}

\begin{lstlisting}
Scenario: one name
  Given a MoinSayer object  # dodgy!  implementation detail...
  When I say "moin" to "bob"
  Then I should see the message "moin bob"

Scenario: say moin to multiple users
  Given a MoinSayer object  # duplicated information
  When I say "moin" to "<user>"
  Then I see the message "<message>"
  Examples:
    | user  | message    |
    | bob   | moin bob   |
    | alice | moin alice |
    | eve   | moin eve   |
\end{lstlisting}

\end{frame}

%%%%%%%%%%%%%%%%%%%%%%%%%%%%%%%%%%%%%%%%%%%%%%%%%%%%%%%%%%%%%%%%%%%%%%%%%%%%
\begin{frame}[fragile]
\frametitle{Background information}

\begin{lstlisting}
Background:
  Given a MoinSayer object  # dodgy!  implementation detail...

Scenario: one name
  When I say "moin" to "bob"
  Then I should see the message "moin bob"

Scenario: say moin to multiple users
  When I say "moin" to "<user>"
  Then I see the message "<message>"
  Examples:
    | user  | message    |
    | bob   | moin bob   |
    | alice | moin alice |
    | eve   | moin eve   |
\end{lstlisting}

\end{frame}

%%%%%%%%%%%%%%%%%%%%%%%%%%%%%%%%%%%%%%%%%%%%%%%%%%%%%%%%%%%%%%%%%%%%%%%%%%%%
\begin{frame}[fragile]
    \frametitle{Comments}

    \begin{itemize}
        \item Comments begin with the \ttt{\#} character
    \end{itemize}
\end{frame}

%%%%%%%%%%%%%%%%%%%%%%%%%%%%%%%%%%%%%%%%%%%%%%%%%%%%%%%%%%%%%%%%%%%%%%%%%%%%
\begin{frame}[fragile]
\frametitle{Tags}

\begin{itemize}
    \item Features and Scenarios can also be tagged
\end{itemize}

\begin{lstlisting}
    @wip
    Scenario: say moin to English speakers
        Given the language "english"
        When I say "moin to "bob"
        Then I should see the message "hello bob"
\end{lstlisting}

\begin{itemize}
    \item ... and then run separately
\end{itemize}
\begin{lstlisting}
$ pherkin -l -t @wip
\end{lstlisting}

\begin{itemize}
    \item ... or skipped
\end{itemize}
\begin{lstlisting}
$ pherkin -l -t ~@wip
\end{lstlisting}

\end{frame}

%%%%%%%%%%%%%%%%%%%%%%%%%%%%%%%%%%%%%%%%%%%%%%%%%%%%%%%%%%%%%%%%%%%%%%%%%%%%
\begin{frame}[fragile]
\frametitle{Pystrings}

\begin{itemize}
    \item Multi-line input can be entered as a \emph{pystring}
\end{itemize}

\begin{lstlisting}
  Scenario: long messages to users
    When the greeting text to "bob" contains
    """
    Greetings, oh most amazing
    and illustrious aquaiantance
    """
    Then the message should be
    """
    Greetings, oh most amazing
    and illustrious aquaintance,
    bob
    """
\end{lstlisting}

\begin{itemize}
    \item Leading whitespace before pystring is trimmed
\end{itemize}

\end{frame}

%%%%%%%%%%%%%%%%%%%%%%%%%%%%%%%%%%%%%%%%%%%%%%%%%%%%%%%%%%%%%%%%%%%%%%%%%%%%
\begin{frame}
\frametitle{Thank you!}

\vspace*{2cm}
\begin{center}
    \LARGE
Thank you for your attention \ttt{:-)}
\end{center}

\end{frame}

\end{document}

% vim: expandtab shiftwidth=4:
