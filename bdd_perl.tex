\documentclass[10pt]{vcs_beamer}

\newcommand{\pending}[1]{\color{DarkOrange}{#1}}
\newcommand{\inscenario}[1]{\color{RoyalBlue}{#1}}
\newcommand{\str}[1]{\color{Turquoise}{#1}}
\newcommand{\error}[1]{\color{red}{#1}}
\newcommand{\pass}[1]{\color{ForestGreen}{#1}}
\newcommand{\ccomment}[1]{\color{DarkBlue}{#1}}

\newcommand{\Feature}{\color{BlueViolet}{Feature:}}
\newcommand{\feature}[1]{\color{BlueViolet}{#1}}

\newcommand{\Background}{\color{BlueViolet}{Background:}}

\newcommand{\Scenario}{\color{BlueViolet}{Scenario:}}
\newcommand{\scenario}[1]{\color{BlueViolet}{#1}}

\newcommand{\ScenarioOutline}{\color{BlueViolet}{Scenario Outline:}}

\newcommand{\Exampless}{\color{BlueViolet}{Examples:}}

\newcommand{\Given}{\color{DarkOrange}{Given}}
\newcommand{\given}[1]{\color{DarkOrange}{#1}}

\newcommand{\When}{\color{RoyalBlue}{When}}
\newcommand{\when}[1]{\color{RoyalBlue}{#1}}

\newcommand{\Then}{\color{ForestGreen}{Then}}
\newcommand{\then}[1]{\color{ForestGreen}{#1}}

\begin{document}

\author{Paul Cochrane}
\title{Adding pherkin to your diet}
\subtitle{Behaviour Driven Development in Perl}
\date{\today}

\maketitle

%%%%%%%%%%%%%%%%%%%%%%%%%%%%%%%%%%%%%%%%%%%%%%%%%%%%%%%%%%%%%%%%%%%%%%%%%%%%
\begin{frame}{What is Behaviour Driven Development?}

\begin{itemize}
    \item Comes from the Agile concept of \emph{User Stories}
    \item A form of requirements specification
    \item Allows business stakeholders to define features in the language of
        their domain
    \item Automated acceptance testing
    \item Outside-in development
    \item Complements and uses TDD
    \item Not a replacement for TDD!
\end{itemize}

\end{frame}

%%%%%%%%%%%%%%%%%%%%%%%%%%%%%%%%%%%%%%%%%%%%%%%%%%%%%%%%%%%%%%%%%%%%%%%%%%%%
\begin{frame}[fragile]
\frametitle{What is BDD? (cont.)}

\usetikzlibrary{positioning}
\usetikzlibrary{arrows}
\begin{figure}
    \begin{tikzpicture}
        [
            all/.style={%
                rectangle,
                thick,
                align=center,
                text width=5cm,
                inner sep=1em,
                outer sep=0pt,
                node distance=1pt,
            },
            user-stories/.style={%
                draw=RoyalBlue,
                fill=RoyalBlue!20,
            },
            high-level/.style={%
                draw=DarkOrange,
                fill=DarkOrange!20,
            },
            low-level/.style={%
                draw=red,
                fill=red!20,
            },
        ]

        \node [all, user-stories] (userStories)
            {User stories};
        \node [all, high-level, below=of userStories] (highLevel)
            {High-level features};
        \node [all, low-level, below=of highLevel] (lowLevel)
            {Low-level functionality};

        \draw [->, thick]
            ($(node cs:name=userStories,anchor=north west) + (-0.5,0)$)
            --
            ($(node cs:name=lowLevel,anchor=south west) + (-0.5,0)$) {};
        \node at ($(node cs:name=userStories,anchor=north west) + (-0.5,0.5)$) {BDD};

        \draw [<-, thick]
            ($(node cs:name=highLevel,anchor=north east) + (0.5,0)$)
            --
            ($(node cs:name=lowLevel,anchor=south east) + (0.5,0)$) {};
        \node at ($(node cs:name=lowLevel,anchor=south east) + (0.5,-0.5)$) {TDD};
    \end{tikzpicture}
\end{figure}

\vspace*{-0.5cm}

Good (and quick) intro to BDD with Cucumber:
\begin{itemize}
    \item Ben Mabey:
        {\footnotesize
        \url{http://www.confreaks.com/videos/72-mwrc2009-bdd-with-cucumber}
        }
\end{itemize}

\end{frame}

%%%%%%%%%%%%%%%%%%%%%%%%%%%%%%%%%%%%%%%%%%%%%%%%%%%%%%%%%%%%%%%%%%%%%%%%%%%%
\begin{frame}[fragile]
    \frametitle{Cucumber}

\begin{itemize}
    \item Stakeholders/Domain experts write requirements in business-readable
        domain-specific language.
    \item This can be parsed and the underlying (matching) tests get executed.
    \item Serves as documentation for the desired features.
    \item Domain experts don't have to write code!
    \item Domain experts \emph{and} developers understand what is to be built.
    \item Parseable natural language: \emph{Gherkin}
\end{itemize}

\begin{lstlisting}[escapeinside='']
'\Feature' User login
    As a user of the system
    In order to see my account information
    I want to be able to log in

    '\Scenario' Successful login
        '\Given' I have an account on the system
        '\When' I enter my username and password
        '\Then' I should be presented with my account information
\end{lstlisting}
\end{frame}

%%%%%%%%%%%%%%%%%%%%%%%%%%%%%%%%%%%%%%%%%%%%%%%%%%%%%%%%%%%%%%%%%%%%%%%%%%%%
\begin{frame}[fragile]
\frametitle{Gherkin the language}

\begin{lstlisting}[escapeinside='']
'\Feature' name of current feature
    '\str{\# description of feature; purely for documentation}'
    In order to [business value]
    As a [role]
    I want to [some action]

    '\str{\# possible background information, relevant to all scenarios}'
    '\Background'
        '\Given' [context/state]

    '\str{\# scenario(s) describing details of feature}'
    '\Scenario' name of current scenario
        '\Given' [context/state]
        '\When' [action]
        '\Then' [outcome]

    '\str{\# also "Scenario Outline:", implemented currently as "Scenario:"}'
    '\Scenario' name of scenario outline
        '\Given', '\When', '\Then', etc.

        '\Exampless'
            | item header | value header |
            | item        | value        |
\end{lstlisting}
\end{frame}

%%%%%%%%%%%%%%%%%%%%%%%%%%%%%%%%%%%%%%%%%%%%%%%%%%%%%%%%%%%%%%%%%%%%%%%%%%%%
\begin{frame}[fragile]
\frametitle{Gherkin works in \emph{your} language}

\begin{lstlisting}[escapeinside='']
'\feature{Funktionalität:}' dickes Auto fahren
    Als deutsche Autofahrer
    Damit ich schnell fahren kann
    Will ich ein dickes Auto haben

    '\scenario{Szenario:}' auf der Autobahn fahren
        '\given{Angenommen}', dass ich auf der Autobahn fahre
        '\when{Wenn}' ich Gas gebe
        '\then{Dann}' soll ich mit mindestens 180 Sachen unterwegs sein
\end{lstlisting}


\begin{lstlisting}[escapeinside='']
'\feature{OHAI:}' I CAN HAS WISHEZ PLZ
    SO DAT I CAN HAS HUGS TIEM AN PLAY TIEM AN SLEEP TIEM
    AS NICE GREEDEE KITTEH
    I CAN PLZ MANEGE TEH WISHEZ

    '\scenario{MISHUN:}' MAK NEW WISH
        '\given{GIVN}' I AM LOGGD IN
        '\when{WEN}' I MAK "CHEEZBURGER" WISH
        '\then{DEN}' I C "CHEEZBURGER" ON ME WISHEZ LIST
\end{lstlisting}

\end{frame}

%%%%%%%%%%%%%%%%%%%%%%%%%%%%%%%%%%%%%%%%%%%%%%%%%%%%%%%%%%%%%%%%%%%%%%%%%%%%
\begin{frame}
    \frametitle{Cucumber in Perl}

    \ttt{Test::BDD::Cucumber}
    \begin{itemize}
        \item Current version: 0.17
        \item Last release: 01 Dec 2013
        \item Usable and many cucumber features implemented
        \item Command line tool: \ttt{pherkin}
            \begin{itemize}
                \item \ttt{pherkin -l} = \ttt{pherkin -I lib}
            \end{itemize}
    \end{itemize}

    Alternative: \ttt{Test::Pcuke}
    \begin{itemize}
        \item Current version: 0.000006
        \item Last release: 03 Dec 2011
    \end{itemize}

\end{frame}

%%%%%%%%%%%%%%%%%%%%%%%%%%%%%%%%%%%%%%%%%%%%%%%%%%%%%%%%%%%%%%%%%%%%%%%%%%%%
\begin{frame}{BDD development cycle}
    \begin{itemize}
        \item Start a feature
        \item Write a scenario describing an aspect of that feature
        \item Write the steps associated with that scenario
            \begin{itemize}
                \item Each \ttt{\Given}, \ttt{\When}, \ttt{\Then} is a \emph{step}
            \end{itemize}
        \item Run \ttt{pherkin}
        \item Steps are undefined
        \item Write automation code in unit-testing framework
        \item Steps fail
        \item \textcolor{red}{Red} $\rightarrow$ \textcolor{green}{Green}
            $\rightarrow$ \textcolor{blue}{Refactor} until step passes
        \item Move on to next step
        \item Wash, rinse, repeat
    \end{itemize}
\end{frame}

%%%%%%%%%%%%%%%%%%%%%%%%%%%%%%%%%%%%%%%%%%%%%%%%%%%%%%%%%%%%%%%%%%%%%%%%%%%%
\begin{frame}[fragile]
\frametitle{Required directory structure for Cucumber}

\begin{lstlisting}
project/
    features/
        my_cool_feature.feature
        step_definitions/
            cool_steps.pl
            other_steps.pl
\end{lstlisting}

\begin{itemize}
    \item \ttt{.feature} file is written in Gherkin
    \item step definition files are written in Perl
    \item step definition files use your favourite testing framework\\
        (e.g. \ttt{Test::More})
    \item Recommended minimum Perl code in a step definition file:
\begin{lstlisting}[language=perl]
use strict;
use warnings;

use Test::More;
use Test::BDD::Cucumber::StepFile;
use Method::Signatures;
\end{lstlisting}
\end{itemize}

\end{frame}

%%%%%%%%%%%%%%%%%%%%%%%%%%%%%%%%%%%%%%%%%%%%%%%%%%%%%%%%%%%%%%%%%%%%%%%%%%%%
\begin{frame}[fragile]
\frametitle{A program to say ``moin''}

Run \ttt{pherkin}
\begin{lstlisting}
$ pherkin
No feature files found at ...
\end{lstlisting}
%stopzone

Create a feature file and a \ttt{step\_definitions/} directory
\begin{lstlisting}[escapeinside='']
$ mkdir -p features/step_definitions
$ vim features/say_moin.feature
'\Feature' Say moin to names in list
    As a user of the software
    In order to be a nice person
    I want to say '\str{"moin"}' to others

    '\Scenario' one name
        '\Given' the name '\str{"bob"}'
        '\When' I say moin
        '\Then' I should see '\str{"moin bob"}'
\end{lstlisting}
%stopzone

\end{frame}

%%%%%%%%%%%%%%%%%%%%%%%%%%%%%%%%%%%%%%%%%%%%%%%%%%%%%%%%%%%%%%%%%%%%%%%%%%%%
\begin{frame}[fragile]
\frametitle{Example: A program to say ``moin''}

Running \ttt{pherkin} again should now show us boilerplate code we can copy
and paste to create our step definitions (standard behaviour of Cucumber and
Behave), however this isn't yet implemented.

\begin{itemize}
    \item Running \ttt{pherkin} shows us that all steps are pending.
\end{itemize}

\begin{lstlisting}[escapeinside='']
Scenario: '\color{RoyalBlue}{one name}'
  '\color{DarkOrange}{Given the name} \color{Turquoise}{"bob"}'
  '\color{DarkOrange}{When I say moin}'
  '\color{DarkOrange}{Then I should see} \color{Turquoise}{"moin bob"}'
\end{lstlisting}

\end{frame}

%%%%%%%%%%%%%%%%%%%%%%%%%%%%%%%%%%%%%%%%%%%%%%%%%%%%%%%%%%%%%%%%%%%%%%%%%%%%
\begin{frame}[fragile]
\frametitle{Example: A program to say ``moin''}

\begin{itemize}
    \item Create the step file
\end{itemize}

\begin{lstlisting}[language=perl]
use warnings;
use strict;

use Test::More;
use Test::BDD::Cucumber::StepFile;
use Method::Signatures;

Given qr//, func($c) {  # $c is an object holding context info
    ok(0);
};

When qr//, func($c) {
    ok(0);
};

Then qr//, func($c) {
    ok(0);
};
\end{lstlisting}
%stopzone

\end{frame}

%%%%%%%%%%%%%%%%%%%%%%%%%%%%%%%%%%%%%%%%%%%%%%%%%%%%%%%%%%%%%%%%%%%%%%%%%%%%
\begin{frame}[fragile]
\frametitle{Example: A program to say ``moin''}

\begin{itemize}
    \item \ttt{Given} step will fail.  \ttt{When} and \ttt{Then} steps are
        still pending.
\end{itemize}

\begin{lstlisting}[escapeinside='']
Scenario: '\inscenario{one name}'
    '\error{Given the name}' '\str{"bob"}'
    '\error{step defined at features/say\_moin.feature line 7.}'
    '\error{ok 1 - Starting to execute step: the name "bob"}'
    '\error{not ok 2}'
    '\error{\#   Failed test at features/step\_definitions/say\_moin\_steps.pl line 16.}'
    '\error{1..2}'
  '\pending{When I say moin}'
  '\pending{Then I should see}' '\str{"moin bob"}'
\end{lstlisting}

\begin{itemize}
    \item Write \ttt{Given} test.
\end{itemize}

\begin{lstlisting}[language=perl]
Given qr/the name "(\w+)"/, func($c) {
    use MoinSayer;
    my $sayer = MoinSayer->new(name => $1);
    ok( $sayer, "Object created" );
    $c->stash->{'scenario'}->{'object'} = $sayer;
};
\end{lstlisting}

\end{frame}

%%%%%%%%%%%%%%%%%%%%%%%%%%%%%%%%%%%%%%%%%%%%%%%%%%%%%%%%%%%%%%%%%%%%%%%%%%%%
\begin{frame}[fragile]
\frametitle{Example: A program to say ``moin''}

\begin{itemize}
    \item \ttt{Given} step naturally still fails (but now for a different
        reason).
    \item Create \ttt{lib} dir, add \ttt{MoinSayer} class and \ttt{name}
        attribute.
\end{itemize}

\begin{lstlisting}[language=perl]
$ mkdir lib
$ vim lib/MoinSayer.pm
...
has 'name' => (
    is => 'ro',
    isa => 'Str',
);
...
\end{lstlisting}
%stopzone

\end{frame}

%%%%%%%%%%%%%%%%%%%%%%%%%%%%%%%%%%%%%%%%%%%%%%%%%%%%%%%%%%%%%%%%%%%%%%%%%%%%
\begin{frame}[fragile]
\frametitle{Example: A program to say ``moin''}

\begin{itemize}
    \item \ttt{Given} step now passes; \ttt{When} step now fails; \ttt{Then}
        step still pending.
\end{itemize}

\begin{lstlisting}[escapeinside='']
$ pherkin -I lib
Scenario: '\inscenario{one name}'
  '\pass{Given the name} \str{"bob"}'
  '\error{When I say moin}'
    '\error{step defined at features/say\_moin.feature line 8.}'
    '\error{ok 1 - Starting to execute step: I say moin}'
    '\error{not ok 2}'
    '\error{\#   Failed test at features/step\_definitions/say\_moin\_steps.pl line 23.}'
    '\error{1..2}'
  '\pending{Then I should see} \str{"moin bob"}'
\end{lstlisting}
%stopzone

\begin{itemize}
    \item \ttt{pherkin -l} is the same as \ttt{pherkin -I lib}
\end{itemize}
\end{frame}

%%%%%%%%%%%%%%%%%%%%%%%%%%%%%%%%%%%%%%%%%%%%%%%%%%%%%%%%%%%%%%%%%%%%%%%%%%%%
\begin{frame}[fragile]
\frametitle{Example: A program to say ``moin''}

\begin{itemize}
    \item Write \ttt{When} test.
\end{itemize}

\begin{lstlisting}[language=perl]
...
my $stdout = capture_stdout {
    $c->stash->{'scenario'}->{'object'}->moin();
};
$c->stash->{'scenario'}->{'stdout'} = $stdout;
...
\end{lstlisting}

\begin{itemize}
    \item Write stub \ttt{moin()} method.
\end{itemize}

\begin{lstlisting}[language=perl]
...
sub moin {
    my $self = shift;
}
...
\end{lstlisting}
%stopzone

\end{frame}

%%%%%%%%%%%%%%%%%%%%%%%%%%%%%%%%%%%%%%%%%%%%%%%%%%%%%%%%%%%%%%%%%%%%%%%%%%%%
\begin{frame}[fragile]
\frametitle{Example: A program to say ``moin''}

\begin{itemize}
    \item \ttt{When} step now passes.  \ttt{Then} step now fails.
\end{itemize}

\begin{lstlisting}[escapeinside='']
Scenario: '\inscenario{one name}'
  '\pass{Given the name} \str{"bob"}'
  '\pass{When I say moin}'
  '\error{Then I should see} \str{"moin bob"}'
    '\error{step defined at features/say\_moin.feature line 9.}'
    '\error{ok 1 - Starting to execute step: I should see "moin bob"}'
    '\error{not ok 2}'
    '\error{\#   Failed test at features/step\_definitions/say\_moin\_steps.pl line 28.}'
    '\error{1..2}'
\end{lstlisting}

\end{frame}

%%%%%%%%%%%%%%%%%%%%%%%%%%%%%%%%%%%%%%%%%%%%%%%%%%%%%%%%%%%%%%%%%%%%%%%%%%%%
\begin{frame}[fragile]
\frametitle{Example: A program to say ``moin''}

\begin{itemize}
    \item Implement \ttt{Then} test and \ttt{moin} sub.
\end{itemize}

\begin{lstlisting}[language=perl]
...
my $stdout = $c->stash->{'scenario'}->{'stdout'};
like( $stdout, qr/moin $1/, "Said expected greeting" );
...
\end{lstlisting}

\begin{lstlisting}[language=perl]
...
sub moin {
    my $self = shift;
    print "moin ", $self->name, "\n";
}
...
\end{lstlisting}
%stopzone

\end{frame}

%%%%%%%%%%%%%%%%%%%%%%%%%%%%%%%%%%%%%%%%%%%%%%%%%%%%%%%%%%%%%%%%%%%%%%%%%%%%
\begin{frame}[fragile]
\frametitle{Example: A program to say ``moin''}

\begin{itemize}
    \item All steps now pass.
\end{itemize}

\begin{lstlisting}[escapeinside='']
Scenario: '\inscenario{one name}'
  '\pass{Given the name} \str{"bob"}'
  '\pass{When I say moin}'
  '\pass{Then I should see} \str{"moin bob"}'
\end{lstlisting}

\begin{itemize}
    \item Now add next scenario and slowly complete the feature
\end{itemize}

\end{frame}

%%%%%%%%%%%%%%%%%%%%%%%%%%%%%%%%%%%%%%%%%%%%%%%%%%%%%%%%%%%%%%%%%%%%%%%%%%%%
\begin{frame}[fragile]
\frametitle{Scenario Outlines}

\begin{itemize}
    \item How to handle multiple inputs and outputs?
\end{itemize}

Will be called \ttt{\ScenarioOutline} in a future version in order to be
compatible with Cucumber.  See Bugs section of README.pod in distro.

\begin{lstlisting}[escapeinside='']
'\Scenario' say moin to multiple users
  '\Given' a MoinSayer object
  '\When' I say "moin" to a "<user>"
  '\Then' I see the message "<message>"

  '\Exampless'
    | user  | message    |
    | bob   | moin bob   |
    | alice | moin alice |
    | eve   | moin eve   |
\end{lstlisting}

\end{frame}

%%%%%%%%%%%%%%%%%%%%%%%%%%%%%%%%%%%%%%%%%%%%%%%%%%%%%%%%%%%%%%%%%%%%%%%%%%%%
\begin{frame}[fragile]
\frametitle{Background information}

\begin{itemize}
    \item Sometimes scenarios have redundant information
    \item This can be extracted into \ttt{Background} information
\end{itemize}

\begin{lstlisting}
Scenario: one name
  # dodgy!  implementation detail...
  Given a MoinSayer object
  When I say "moin" to "bob"
  Then I should see the message "moin bob"

Scenario: say moin to multiple users
  # duplicated information
  Given a MoinSayer object
  When I say "moin" to "<user>"
  Then I see the message "<message>"
  Examples:
    | user  | message    |
    | bob   | moin bob   |
    | alice | moin alice |
    | eve   | moin eve   |
\end{lstlisting}

\end{frame}

%%%%%%%%%%%%%%%%%%%%%%%%%%%%%%%%%%%%%%%%%%%%%%%%%%%%%%%%%%%%%%%%%%%%%%%%%%%%
\begin{frame}[fragile]
\frametitle{Background information}

\begin{lstlisting}
Background:
  # dodgy!  implementation detail...
  Given a MoinSayer object

Scenario: one name
  When I say "moin" to "bob"
  Then I should see the message "moin bob"

Scenario: say moin to multiple users
  When I say "moin" to "<user>"
  Then I see the message "<message>"
  Examples:
    | user  | message    |
    | bob   | moin bob   |
    | alice | moin alice |
    | eve   | moin eve   |
\end{lstlisting}

\end{frame}

%%%%%%%%%%%%%%%%%%%%%%%%%%%%%%%%%%%%%%%%%%%%%%%%%%%%%%%%%%%%%%%%%%%%%%%%%%%%
\begin{frame}[fragile]
    \frametitle{Comments}

    \begin{itemize}
        \item Comments begin with the \ttt{\#} character
        \item Must be alone on a line
    \end{itemize}

    Correct:
    \begin{lstlisting}[escapeinside='']
    '\ccomment{\# some meta-information about the following step}'
    '\Given' blah
    '\When' moo
    '\Then' baa
    \end{lstlisting}


    Incorrect:
    \begin{lstlisting}[escapeinside='']
    '\Given' blah # what does blah mean?
    '\When' moo
    '\Then' baa
    \end{lstlisting}
\end{frame}

%%%%%%%%%%%%%%%%%%%%%%%%%%%%%%%%%%%%%%%%%%%%%%%%%%%%%%%%%%%%%%%%%%%%%%%%%%%%
\begin{frame}[fragile]
\frametitle{Tags}

\begin{itemize}
    \item Features and Scenarios can also be tagged
\end{itemize}

\begin{lstlisting}
    @wip
    Scenario: say moin to English speakers
        Given the language "english"
        When I say "moin to "bob"
        Then I should see the message "hello bob"
\end{lstlisting}

\begin{itemize}
    \item ... and then run separately
\end{itemize}
\begin{lstlisting}
$ pherkin -l -t @wip
\end{lstlisting}

\begin{itemize}
    \item ... or skipped
\end{itemize}
\begin{lstlisting}
$ pherkin -l -t ~@wip
\end{lstlisting}

\end{frame}

%%%%%%%%%%%%%%%%%%%%%%%%%%%%%%%%%%%%%%%%%%%%%%%%%%%%%%%%%%%%%%%%%%%%%%%%%%%%
\begin{frame}[fragile]
\frametitle{Pystrings}

\begin{itemize}
    \item Multi-line input can be entered as a \emph{pystring}
\end{itemize}

\begin{lstlisting}
  Scenario: long messages to users
    When the greeting text to "bob" contains
    """
    Greetings, oh most amazing
    and illustrious aquaiantance
    """
    Then the message should be
    """
    Greetings, oh most amazing
    and illustrious aquaintance,
    bob
    """
\end{lstlisting}

\begin{itemize}
    \item Leading whitespace before pystring is trimmed
\end{itemize}

\end{frame}

%%%%%%%%%%%%%%%%%%%%%%%%%%%%%%%%%%%%%%%%%%%%%%%%%%%%%%%%%%%%%%%%%%%%%%%%%%%%
\begin{frame}
    \frametitle{CI server integration}
    Run your Cucumber steps as part of your test suite.
\end{frame}

%%%%%%%%%%%%%%%%%%%%%%%%%%%%%%%%%%%%%%%%%%%%%%%%%%%%%%%%%%%%%%%%%%%%%%%%%%%%
\begin{frame}
    \frametitle{More Cucumber features}
    Transform, Begin, After, About, Hooks
\end{frame}

%%%%%%%%%%%%%%%%%%%%%%%%%%%%%%%%%%%%%%%%%%%%%%%%%%%%%%%%%%%%%%%%%%%%%%%%%%%%
\begin{frame}
\frametitle{Thank you!}

\vspace*{2cm}
\begin{center}
    \LARGE
Thank you for your attention \ttt{:-)}
\end{center}

\end{frame}

\end{document}

% vim: expandtab shiftwidth=4:
